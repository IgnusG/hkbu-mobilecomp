\documentclass[12pt]{article}

\usepackage[dvipsnames]{xcolor} % Colors
\usepackage{amsmath} % Math
\usepackage{witharrows} % Equation Arrows
\usepackage[normalem]{ulem} % Strike-through Text
\usepackage{graphicx} % Images
\usepackage{titlesec} % Title Configuration
\usepackage{enumitem} % More Configurable enumerate
\usepackage{pifont} % More symbols

\graphicspath{ {./images/} }

% Command to strike-through text in math equations
\newcommand{\cross}[1]{\text{\sout{\ensuremath{#1}}}}
% Adjust the format of subsection (#. )
\renewcommand{\thesubsection}{\arabic{subsection}.\hspace{0.2em}}
% Change subsub sections numbering to alphabetical (a))
\renewcommand{\thesubsubsection}{\thesubsection\alph{subsubsection})}
% Configure sub and subsub sections to display inline
\setcounter{secnumdepth}{3}
\titleformat{\subsection}[runin]
  {\normalfont\normalsize\bfseries}{\thesubsection}{0em}{}
\titleformat{\subsubsection}[runin]
  {\normalfont\normalsize\bfseries}{\thesubsubsection}{0.5em}{}
% Create new commands to reference exercises
\newcommand{\exercise}{\subsection{}\setcounter{subsubsection}{0}}
\newcommand{\multipartexercise}{\addtocounter{subsection}{1}\setcounter{subsubsection}{0}}
\newcommand{\exercisepart}{\subsubsection{}}

\title{
    COMP4097 Mobile Computing \linebreak
    Assignment 2
}
\author{Jonatan Juhas (19502966)}
\date{\today}

\begin{document}
\maketitle

\multipartexercise
\exercisepart
As the hamming code (7 4) requires a message length of 4, we split it into 2 parts: 1000 and 1010.

The first part (assuming the pattern M7 M6 M5 P4 M3 P2 P1) is:

\begin{description}[labelwidth=0pt]
\item[]
\begin{DispWithArrows*}[format=cccccccr,fleqn,mathindent=0pt]
    1 &\ 0 &\ 0 &\ \_ &\ 0 &\ \_ &\ \_ &
    \Arrow{P1: $\_001 \rightarrow 0+0+1=1 \ (odd) \rightarrow 1001$} \\
    1 &\ 0 &\ 0 &\ \_ &\ 0 &\ \_ &\ 1 &
    \Arrow{P2: $\_001 \rightarrow 0+0+1=1 \ (odd) \rightarrow 1001$} \\
    1 &\ 0 &\ 0 &\ \_ &\ 0 &\ 1 &\ 1 &
    \Arrow{P4: $\_101 \rightarrow 1+0+1=2 \ (even) \rightarrow 0101$} \\
    1 &\ 0 &\ 0 &\ 0 &\ 0 &\ 1 &\ 1 &
\end{DispWithArrows*}
\end{description}

The second part is:

\begin{description}[labelwidth=0pt]
\item[]
\begin{DispWithArrows*}[format=cccccccr,fleqn,mathindent=0pt]
    1 &\ 0 &\ 1 &\ \_ &\ 0 &\ \_ &\ \_ &
    \Arrow{P1: $\_001 \rightarrow 0+1+1=2 \ (even) \rightarrow 0011$} \\
    1 &\ 0 &\ 1 &\ \_ &\ 0 &\ \_ &\ 0 &
    \Arrow{P2: $\_001 \rightarrow 0+0+1=1 \ (odd) \rightarrow 1001$} \\
    1 &\ 0 &\ 1 &\ \_ &\ 0 &\ 1 &\ 0 &
    \Arrow{P4: $\_101 \rightarrow 1+0+1=2 \ (even) \rightarrow 0101$} \\
    1 &\ 0 &\ 1 &\ 0 &\ 0 &\ 1 &\ 0 &
\end{DispWithArrows*}
\end{description}

So the message transmitted to Station B will be 1001011 1010010

\exercisepart
We receive 2 messages: 1011011 and 1011010.
For the first message we check all parity bits:
\\
P1: $1011 \rightarrow 1+0+1+1=3 \ \color{Red} \ \text{\ding{55}}$ \\
P2: $1001 \rightarrow 1+0+0+1=2 \ \color{Green} \ \text{\ding{51}}$ \\
P4: $1101 \rightarrow 1+1+0+1=3 \ \color{Red} \ \text{\ding{55}}$

Correcting the message with $\text{P1} + \text{P4} = 5$ to 1001011.

\bigskip

We check the parity bits in the second message:
\\
P1: $0011 \rightarrow 0+0+1+1=2 \ \color{Green} \ \text{\ding{51}}$ \\
P2: $1001 \rightarrow 1+0+0+1=2 \ \color{Green} \ \text{\ding{51}}$ \\
P4: $1101 \rightarrow 1+1+0+1=3 \ \color{Red} \ \text{\ding{55}}$

Correcting the message with $\text{P4} = 4$ to 1010010.

\exercisepart
We again receive 2 messages after the spike: 1100111 and 1011010.
The first message check results in:
\\
P1: $1101 \rightarrow 1+1+0+1=3 \ \color{Red} \ \text{\ding{55}}$ \\
P2: $1111 \rightarrow 1+1+1+1=4 \ \color{Green} \ \text{\ding{51}}$ \\
P4: $0011 \rightarrow 0+0+1+1=2 \ \color{Green} \ \text{\ding{51}}$

Correcting the message with $\text{P1} = 1$ to 1100110.

\bigskip

The second message check:
\\
P1: $0011 \rightarrow 0+0+1+1=2 \ \color{Green} \ \text{\ding{51}}$ \\
P2: $1001 \rightarrow 1+0+0+1=2 \ \color{Green} \ \text{\ding{51}}$ \\
P4: $1101 \rightarrow 1+1+0+1=3 \ \color{Red} \ \text{\ding{55}}$

Correcting the message with $\text{P4} = 4$ to 1010010.

\bigskip

The channel seems to be quite unreliable but we manage to fix all errors thanks to the hamming code encoding.

\end{document}
